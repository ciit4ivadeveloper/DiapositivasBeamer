\documentclass{beamer}

% Tema de la presentación
\usetheme{Madrid}




\begin{document}

\title{Análisis Comparativo de Selenium y Appium para Pruebas de Regresión en la Aplicación Móvil  de la Cooperativa de Ahorro y Crédito Jardín Azuayo  ”JAMOVIL”: Calidad ,Eficiencia, Cobertura y Experiencia del Usuario}
\author{Cristian Ivan Idrovo Tapia, Santiago David Cordero Crespo} 
\date{\today}

\begin{frame}
  \titlepage
\end{frame}



\begin{frame}{Tabla de Contenido}
  \tableofcontents
\end{frame}

\section{Introducción}


\begin{frame}{Introducción}

La constante evolución de aplicaciones móviles ha surgido la necesidad de potenciar herramientas específicas para automatizar las pruebas de regresión en dispositivos móviles tales como:

 \begin{itemize}
    \item Appium.
    \item Selenium.
  \end{itemize} 

A medida que la competencia en el mercado de aplicaciones móviles se intensificó, las pruebas de regresión comenzaron a centrarse más en la experiencia del usuario.
\end{frame}


\section{Justificación}

\begin{frame}{Justificación}
La  aplicación móvil financiera de la cooperativa de ahorro y crédito Jardin Azuayo "JAMOVIL" es una solución tecnologica que brinda una experiencia efeciente y accesible para sus usuarios, facilitando operaciones financieras y transacciones. Garantizar la calidad, eficiencia y experiencia del usuario en este tipo de aplicaciones es fundamental para  mantener una ventaja competitiva en el mercado financiero móvil.

La comparación entre dos herramientas líderes de automatización de pruebas, Selenium y Appium, permitirá identificar cuál de ellas es más adecuada para realizar pruebas de regresión en aplicaciones móviles financieras. 

Esto ayudara a los desarrolladores a acelerar la liberación de nuevas versiones, lo que mejora la eficiencia del equipo de desarrollo, optimizara los tiempos de pruebas de calidad de software y experiencia de usuario
\end{frame}

\section{Objetivos}
\subsection{General}
\begin{frame}{Objetivo General }
  \begin{columns}
    \begin{column}{0.5\textwidth}
     Evaluar y comparar la eficiencia, cobertura y experiencia del usuario al realizar pruebas de regresión en la aplicación móvil financiera "JAMOVIL", utilizando las herramientas de automatización Selenium y Appium, para mejorar la calidad del software y optimizar la eficiencia del proceso de pruebas. 
   
    \end{column}
    \begin{column}{0.5\textwidth}
      \begin{itemize}
          \item Que voy hacer ?
          \begin{itemize}
              \item Evaluar y comparar la eficiencia, cobertura y experiencia del usuario al realizar pruebas de regresión en la aplicación móvil financiera "JAMOVIL"
          \end{itemize}
          \item Como lo voy hacer ?
          \begin{itemize}
              \item Utilizando las herramientas de automatización Selenium y Appium
          \end{itemize}
          \item ¿Para qué lo voy a hacer?
           \begin{itemize}
              \item Para mejorar la calidad del software y optimizar la eficiencia del proceso de pruebas 
          \end{itemize}         
          \end{itemize}
    \end{column}
  \end{columns}
\end{frame}

\subsection{Objetivo Específico 1 }
\begin{frame}{Objetivo Específico 1 }
  \begin{columns}
    \begin{column}{0.5\textwidth}
      Realizar una revisión exhaustiva de la literatura científica y técnica sobre pruebas de regresión en aplicaciones móviles financieras, enfocándose en estudios, investigaciones para identificar y analizar las herramientas de automatización Selenium y Appium
    \end{column}
    \begin{column}{0.55\textwidth}
      \begin{itemize}
          \item Que voy hacer ?
          \begin{itemize}
              \item Realizar una revisión exhaustiva de la literatura científica y técnica sobre pruebas de regresión en aplicaciones móviles financieras
          \end{itemize}
          \item Como lo voy hacer ?
          \begin{itemize}
              \item enfocándose en estudios, investigaciones 
          \end{itemize}
          \item ¿Para qué lo voy a hacer?
          \begin{itemize}
              \item para identificar y analizar las herramientas de automatización  Selenium y Appium 
          \end{itemize}
      \end{itemize}
    \end{column}
  \end{columns}
\end{frame}
\subsection{Objetivo Específico 2 }
\begin{frame}{Objetivo Específico 2 }
  \begin{columns}
    \begin{column}{0.5\textwidth}
     Definir el conjunto de requerimientos y criterios de evaluación utilizando herramientas de automatización Selenium y Appium en las pruebas de regresión de la aplicación móvil financiera "JAMOVIL" para evaluar y comparar la eficiencia, calidad, cobertura de casos de prueba y experiencia del usuario 
    \end{column}
    \begin{column}{0.5\textwidth}
      \begin{itemize}
          \item Que voy hacer ?
          \begin{itemize}
              \item Definir el conjunto de requerimientos y criterios de evaluación
          \end{itemize}
          \item Como lo voy hacer ?
          \begin{itemize}
              \item utilizando herramientas de automatización Selenium y Appium en las pruebas de regresión de la aplicación móvil financiera "JAMOVIL"
          \end{itemize}
          \item ¿Para qué lo voy a hacer?
          \begin{itemize}
                  \item utilizando herramientas de automatización Selenium y Appium en las pruebas de regresión de la aplicación móvil financiera "JAMOVIL"
          \end{itemize}
      \end{itemize}
    \end{column}
  \end{columns}
\end{frame}
\subsection{Objetivo Específico 3}
\begin{frame}{Objetivo Específico 3 }
  \begin{columns}
    \begin{column}{0.5\textwidth}
      Implementar la automatización de pruebas con Selenium y Appium, asegurando la configuración adecuada en el ambiente de pruebas y dispositivos móviles, para recolectar información que será utilizada en el proceso de análisis. 
    \end{column}
    \begin{column}{0.5\textwidth}
      \begin{itemize}
          \item Que voy hacer ?
          \begin{itemize}
              \item  Implementar la automatización de pruebas con Selenium y Appium.
          \end{itemize}
          \item Como lo voy hacer ?
          \begin{itemize}
              \item Asegurando la configuración adecuada en el ambiente de pruebas y dispositivos móviles.
          \end{itemize}
         \item ¿Para qué lo voy a hacer?
         \begin{itemize}
              \item Para recolectar información que será utilizada en el proceso de análisis.
          \end{itemize}
      \end{itemize}
    \end{column}
  \end{columns}
\end{frame}

\subsection{Objetivo Específico 4}
\begin{frame}{Objetivo Específico 4 }
  \begin{columns}
    \begin{column}{0.5\textwidth}
      Medir y comparar las pruebas de regresión obtenida con Selenium y Appium, considerando el número de casos de prueba ejecutados y la profundidad de la cobertura en diferentes escenarios de prueba, para determinar cual de las dos herramientas ofrece una mayor cobertura de pruebas. 
    \end{column}
    \begin{column}{0.5\textwidth}
      \begin{itemize}
          \item Que voy hacer ?
          \begin{itemize}
              \item  Medir y comparar las pruebas de regresión obtenida con Selenium y Appium
          \end{itemize}
          \item Como lo voy hacer ?
          \begin{itemize}
              \item considerando el número de casos de prueba ejecutados y la profundidad de la cobertura en diferentes escenarios de prueba
          \end{itemize}
         \item ¿Para qué lo voy a hacer?
         \begin{itemize}
              \item Para determinar cual de las dos herramientas ofrece una mayor cobertura de pruebas
          \end{itemize}
      \end{itemize}
    \end{column}
  \end{columns}
\end{frame}
\subsection{Objetivo Específico 5}
\begin{frame}{Objetivo Específico 5 }
  \begin{columns}
    \begin{column}{0.5\textwidth}
      Identificar fortalezas y debilidades de cada herramienta de automatización Selenium y Appium, considerando factores como la complejidad de la implementación, la facilidad de mantenimiento y la escalabilidad, para tomar las decision sobre cual herramienta es la mas adecuada para realizar el proceso de pruebas en la JAMOVIL 
 
    \end{column}
    \begin{column}{0.5\textwidth}
      \begin{itemize}
          \item Que voy hacer ?
          \begin{itemize}
              \item Identificar fortalezas y debilidades de cada herramienta de automatización Selenium y Appium
          \end{itemize}
          \item Como lo voy hacer ?
          \begin{itemize}
              \item considerando factores como la complejidad de la implementación, la facilidad de mantenimiento y la escalabilidad
          \end{itemize}
         \item ¿Para qué lo voy a hacer?
         \begin{itemize}
              \item para tomar las decision sobre cual herramienta es la mas adecuada para realizar el proceso de pruebas en la JAMOVIL
          \end{itemize}
      \end{itemize}
    \end{column}
  \end{columns}
\end{frame}



\section{Metodología}

\begin{frame}{Metodología}
Se ejecutará una revisión de la literatura científica y técnica sobre pruebas de regresión en la aplicaciones móvil “JAMOVIL”, enfocándose en estudios relevantes. Posterior a ello, se analizarán y evaluarán las herramientas de automatización Selenium y Appium, identificando sus características, ventajas y desventajas en el contexto específico de pruebas de regresión.
Luego, se definirán requerimientos y criterios de evaluación para las pruebas de regresión de la aplicación "JAMOVIL", considerando aspectos como eficiencia, calidad, cobertura de casos de prueba y experiencia del usuario durante el proceso de pruebas.
La implementación de la automatización de pruebas involucra configurar el ambiente de pruebas con dispositivos móviles, emuladores o simuladores,  para recolectar datos relevantes para el análisis.
En la etapa de medición y comparación de resultados, se ejecutarán pruebas de regresión con ambas herramientas en diferentes escenarios de la aplicación "JAMOVIL". Se compararán los resultados, teniendo en cuenta el número de casos de prueba ejecutados y la profundidad de la cobertura para determinar la herramienta más adecuada. 
\end{frame}



\section{Cronograma}

\begin{frame}{Cronograma}
  \begin{itemize}
    \item<1-> Elemento 1 (Aparece desde la diapositiva 1 en adelante)
    \item<2-> Elemento 2 (Aparece desde la diapositiva 2 en adelante)
    \item<3-> Elemento 3 (Aparece desde la diapositiva 3 en adelante)
    \item<4-> Elemento 4 (Aparece desde la diapositiva 4 en adelante)
  \end{itemize}
\end{frame}

\section{Presupuesto}
\begin{frame}{Presupuesto}
    
\end{frame}

\end{document}

